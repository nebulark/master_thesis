
\section{Sewing Worlds Together With SEAMS}
https://www.cg.tuwien.ac.at/research/vr/seams/index.html
\section{Implementation}
\subsection{Why Vulkan}
Cross plattform with MoltenVK
(Probably) More possibilities due to more explicit compared to OpenGL
More Potential to optimize
(Learn effekt


\subsection{layout(earlyfragmenttests)}

\subsection{Remove Stencil + Add instance drawing}
as we are rendering the object multiple times we could use instance drawing. However this doesn't work if we want to switch stencil inbetween.
we fall back to manual stencil testing (which also helps us to render on nvidia), with almost no performance drop
instanced drawing improved rendering speed portal only instanced 23 -> 18 ms
scene instanced 18 ms -> 8ms
this is huge.
it also allows us to increase stencil buffer bits if needed

\subsection{possible draw indirect improvement}
we repeat ourselves quit a bit, we could improve this with draw indirect
additionally the code that detects portal visibility could conditionally set the instance count to 0 or the right value, which could improve performance drastically

\subsection{Portal Rendering}
1 Draw Scene
2 Render Portals, setting coresponding stencil buffer

3 for each visible Portal

3.1 Move Camera to portal Pos
3.2 Portal Depth texture (set everything to maxdepth)
3.3 Render one Portal store depth Values in Portal depth (coud be paralell to regular scene rendering)
3.4 Render Scene with depth and stencil test (only in portal area). Discard fragments before portal depth.
(Instead of portal depth, a clip plane could be used for planar portals)
3.5 The for each visible portal do 3

endfor

Rekursion? Need to handle it for curve portals. Line could cross portal 2 or more times for curved portals


To Move the camera a matrix is applied, which would move a portal excatly at its partner-Portals position. Scaling??? Would be nice.
When cursing 2 times, the rendering of the first portal is disabled as it would be discarded due to the portal depth texture.

\subsection{Actual Portal rendering}
2 Passes for iteration, scene pass then portal pass
scene pass can contain different pipelines (regular objects, lines .etc)
portal pass uses stencil export and writes stencil and depth values into rendered depth stencil. It also calculates if it is drawn and sets matrix indices accordingly

subsequent passes use rendered depth as additional dicard potential, as well as using stencil and using the matrix index to find correct matrix
subsequene portal pass uses rendered stencil to cut down the amount of portals drawn. 

1 Pass Draw Scene
2 Pass Draw Portal
2 +2 * i Pass draw scene (multiple times for each previously drawn portal)
2 + 2* i + 1 pass draw portals (multiple times for each previously drawn portal)



Maybe use a pre depth only pass? Would reduce the amount of "visible" portals. Maybe could even speed up

\subsection{ Dynamic protals : Discarding in Fragment shader}
stencil check happens after we instantiate the shader, portals count as visible even if they would have been cut out by stencil test. This limits the number of visible portals and we can't use the property that the portal window is so small that only a few portals will be visible.

maybe we can perform a manual stencil test

\subsection{Not Translating portals}
e.g. scaling / mirror ...
render twice once with only front faces and onces with only backfaces as the front and back face each have a different camera matrix and stencil value.
Does only work for ojects that would look the same after the transformation. i.e. any rotation for spheres, 90 degree rotation for cubes, inverting (scalie with - 1)for symetrical objects ...
Proabably only rotation can be used for these. Scaling maybe only for inverting / mirroring. Others won't work.


\subsection{Teleporting}
Dectect collision with portal with raytrace. Old and new pos are the eye position.
Teleport by applying the teleport matrix (matrix with transforms from endpoint a to endpoint b) to camera. Position and Camera matrix are multiplied by this matrix every time. it starts with identity matrix.

Maybe keep an matrix stack, so we can recalculate, allow the perfect cancelation of with inverse matrices. Otherwise we might get numerical differences.

\subsection{Kd Tree Surface area heuristic}
\cite{ macdonald:1990:heuristics}
\subsection{Stencil values}
increasing numbers for each portals?
Reuse previous used values, but clearing stencil greater than value needed to keep?
only one reference value, use read and write mask? Toggle between most/least 4 significan bits? Or use incrementing?
use write mask with invert operation so set specific value if we the value in the stencil buffer.
to compare with 2 and set to 3 we would
readmask FF -> equal to ref value 2
writemask = 2 xor 3 and invert if passed -> we know its to so we have 2 xor 2 xor 3 -> 3

readmask + ref value for test
writemask + invert to write exact value (only for equality tests)

how to write with ref value?

readmask
writemask all + ref value

found no "clever" way to set stencil ref in shader and use it as good compare value
stencil ref value is the compare value + additional value so it is unique for later operations.

so first iteration portals ids are ????XXXX
second iteration use bits from previous -> ??YY XXXX, we can mask the prefix and use that as compare value

number of bits can vary for each iteration. First iterations should have the most bits. Later iterations only render in a smal fraction of the screen so it is unlikely that that many portals exist.
Probably use 4 bits for first iteration to be able to see 16 portals at once
second iteration might also have a high number of bits, so that we cover the case of directly standing in front of portal
third iteration probably doesn need many bits as we can design the level to avoid 2 portals close to each other, so that the screen coverage would be small.

Maybe allow to adjust this dynamically? Probably possible to dectect how many portals could be visible initially. And give the second iteration more bits. If we stand in front of a portal, we only need 1 bit for that iteration but the second one probably could need a lot more, as the screen coverage is bit.
Decting number of portals for further passes seems to difficult and mostly not worth it. But potentially decreasing iteration 1 and giving those to iteration 2 could be really worth it.


portal ids must begin with 1, so that they can always be differentiated from their parents
this effectively reduces the portal count by iteration by 1.

However as no parent can have a zero prefix, we can create new intial portal with a zero prefix, with id starting at 1, however this uses up twice the bits an limits the recursion for that portal. Could be useful if we want to have many portals and we know that some of them won't need to recurse as often.
This can be done again, but limits iteration even further for such portals

example for 2 bit portal ids

initial portals

01 - portal 1
10 - portal 2
11 - portal 3
00 01 - portal 4
00 10 - portal 5
00 11 - portal 6

01 00 XX portal 7-9
10 00 XX portal 10-12
11 00 XX portal 13-15

00 00 XX Portal 16-18

But  maybe there are other usecases?




\subsection{Max portals vs Max visible portals}
instead of using a portal id for stencil ref, we calculate the id by finden how many portals were drawn before the current portal.
To count this we can use a buffer, initially set to 0. Each portals fragment shader writes a 1 to that buffer at its index. If it is never invoked the initial zero is automagically kept. The calculate the value we add all the numbers of the buffer together.
We also need to indirect the camera mats with an index. Drawn portals will write their index for their camera mat into this camera index array.
To find a camera index, look at the correct location in the index array get the index and use it to look up the camera mat in the camera array. a -1 could represent that there is no drawing, due to fewer than max visible portals beeing visible. If its zero we can just set all vertices to the same value, to create a degenerate triangle.
This could maybe be improved with indirect drawcall to save vertex shader ivokations.

\subsection{Dont draw camera in intial scene rendering!!!}

\subsection{ Curved space: Matrix to transform plane into curved plane}
maybe something like this
1 0 0
0 1 0
1 1 1
scales z, depending on x an y

using orhograhic camera?
chane perspective angle


\subsection{skipping self render helper}
\subsection{how to translate camera}
matrix multiplication does not work.
calc vector camera -> PortalA
rotate that vector by aToB rotation
subtract it from PortalB location
-> Translation

Rotation = camera rotation * aToB rotation

\subsection{How to really translate camera using matrices}
Bring the camera into Portal a space by multiplying it with the inverse of the portal a transform
transform the camera using Portal B transform which would move a point to portal Bs loaction. Camera is not a origin, so it will not be at portal b location, but some relative transformation away from Portal B.




\subsection{Portal drawing orderer Depth first vs breadth first}

breadth first seems better as we only need to clear depth buffer after each layer, because
each draw in a layer will never rendert to the same pixel as another draw in that layer
breadth first allows us to reduce pipelinecount if we use dynamic stencil ref and stencil compare mask

breadth first is easier to stop at a layer

depth first might be useful to easier skip portals, but probably only influences code complexity

\subsection{Stencil value generation, bit patterns}
bit pattern analyse

most significan bits parent child num

can generate mask directly if using power of 2 portal count
mask = index + V

V(Layer) = 4^{Layer-1} + V(Layer-1)
V(0) = 0;

\subsection{Subpasses}
Initial idea one subpass for each element in our draw portal tree.
Inital layer: Scene pass and Portal pass
Next layer: for each Portal pass from pervious layer, do a scene + portal pass
-> superlinear increase of Subpasses!!!
-> Specifying dependencies and clears are complicated

New approach
One Scene + Portal pass for each layer
Use dynamic depth stencil state


\subsection{Occlusion Queries}
Use Occulusion queries to selectively render portals? Requires Readback. Could hide latency by delaying frames. Seems difficult.
Alternativly use Draw Indirect to skip drawing (or the conditional drawing extension). Needs to be able to set stencil buffer (shader stencil export) in Shader or use a technique similar to stencil buffer.


\subsection{Software Raytracing}
Is too slow. But only bare. It it were 10 Times faster (before further optimisations) this could have worked.


\subsection{Stencil Buffer}
Valve talk

\subsection{Open CL / Vulkan}
\section{GPU Data Structures}
\subsection{Sparse Data Structure}

\section{General Purpose GPU}

\section{GPU Data Structures}

test \cite{Heuser:2003}