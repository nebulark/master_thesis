%%%%%%%%%%%%%%%%%%%% author.tex %%%%%%%%%%%%%%%%%%%%%%%%%%%%%%%%%%%
%
% sample root file for your "contribution" to a proceedings volume
%
% Use this file as a template for your own input.
%
%%%%%%%%%%%%%%%% Springer %%%%%%%%%%%%%%%%%%%%%%%%%%%%%%%%%%


\documentclass{article}
%
% RECOMMENDED %%%%%%%%%%%%%%%%%%%%%%%%%%%%%%%%%%%%%%%%%%%%%%%%%%%
%

% to typeset URLs, URIs, and DOIs
\usepackage{url}
\usepackage{abbrevs}
\usepackage{graphicx}
\usepackage{dsfont}
\usepackage[backend=biber]{biblatex}
\usepackage{float}

\def\UrlFont{\rmfamily}
\usepackage[acronym]{glossaries}

\usepackage{csquotes}
\bibliography{bibliography}

	
%Term definitions
%\newacronym{sdf}{SDF}{Signed Distance Function}

\newacronym{api}{API}{application programming interface}
\newacronym{cpu}{CPU}{central processing unit}
\newacronym{gpu}{GPU}{graphics processing unit}
\newacronym{ubo}{UBO}{uniform buffer object}
\newacronym{aabb}{AABB}{axis aligned bounding box}
\newacronym{ue4}{UE4}{Unreal Engine 4}
\newacronym{sdl}{SDL}{Simple Directmedia Layer}

\newacronym{amd}{AMD}{Advanced Micro Devices}
\newacronym{glm}{GLM}{OpenGL Mathematics}
\newacronym{gsl}{GSL}{Guidelines Support Library }
\newacronym{fps}{FPS}{frames per second}
\newacronym{vr}{VR}{virtual reality}
\newacronym{pvs}{PVS}{potentially visible set}
\newacronym{psb}{PSB}{portal screen \gls{aabb}}
\newacronym{vfc}{VFC}{view frustum culling}
\newacronym[plural=OBBs,firstplural=oriented bounding boxes (OBBs)]{obb}{OBB}{oriented bounding box}

\newacronym{wim}{WIM}{World in Miniature}
\newacronym{glsl}{GLSL}{OpenGL Shading Language}
 
\newglossaryentry{modelmatix}{name={model matrix}, description={An objects position, rotation and scale represented as matrix}, plural={model matrices}}

\newglossaryentry{cameramatrix}{name={camera matrix}, description={The cameras position and rotation represented as matrix. Other term for the camera's \gls{modelmatix}}.}

\newglossaryentry{viewmatrix}{name={view matrix}, description={A view matrix is used to draw from the viewpoint of the oject the viewmatrix belongs to.}, plural={view matrices}}



\newglossaryentry{portalpair}{name={portal pair}, description={Two connected portals. When travelling through a portal pair's portal the  object is teleported to the  portal pair's other portal.}}

\newglossaryentry{endpoint}{name={endpoint}, description={One portal of a portal pair.}}

\newglossaryentry{teleportationmatrix}{name={teleportation matrix}, description={An endpoint's matrix that when applied to an object moves that object to the other endpoints location}, plural={teleportation matrices}}

\newglossaryentry{renderpass}{name={renderpass}, description={A renderpass defined by Vulkan specification \cite{khronos:vulkan:spec1.1}. Can have multiple subpasses.}}




\newglossaryentry{subpass}{name={subpass}, description={A subpass defined by Vulkan specification \cite{khronos:vulkan:spec1.1}. Belongs to a renderpass.}}

\newglossaryentry{portalset}{name={portal set}, description={The set of all Portals.}}
\newglossaryentry{portalsetid}{name={portal set id}, description={Unique id within a recursion for drawing a \gls{portalset}. Is equal to the number of \glspl{portalset} drawn before the current one in the current recursion.}}


\newglossaryentry{portalcount}{name={portal count}, description={The number of portals within the \gls{portalset}}}
\newglossaryentry{portalid}{name={portal set}, description={An unique integer indentifying a portal.}}

\newglossaryentry{object}{name={object}, description={A entity which can be drawn.}}
\newglossaryentry{objectset}{name={object set}, description={The set of all Objects}}

\newglossaryentry{recursioncount}{name={recursion count}, description={How often the view can traverse through portals.}}


\newglossaryentry{viewray}{name={view ray}, description={A ray that is cast for raytring. Its intersection determines a pixel's color.}}






\usepackage{listings,xcolor}
\lstset{language=C++}
\lstset{
	basicstyle=\footnotesize\ttfamily,
	keywordstyle=\bfseries\color{blue},
	breaklines=true,
	tabsize=2,
	alsoother={\#},
	deletekeywords ={\#pragma},
	deletedelim=*[directive]\#,
	morekeywords={mat4, vec3, vec4, in, out, discard, layout, push\_constant, uniform, uint}
}

%%\lstset{
%%	emph={mat4, vec3, vec3, in, out},emphstyle={\color{blue}}%
%%	emph={discard},emphstyle={\color{blue}}%
%}%


\begin{document}


\begin{abstract}
	% !TeX spellcheck = en_GB

%The abstract should summarize the contents of the paper
%using at least 70 and at most 150 words. It will be set in 9-point
%font size and be inset 1.0 cm from the right and left margins.
%There will be two blank lines before and after the Abstract. \dots

Ray tracing is a rendering technique, which allows to generate more photorealistic images, when compared with rasterization techniques. Nevertheless, this comes with the cost of a much greater computational expense. However, ray tracing allows rendering objects, which are represented by other means than explicit vertex data \cite{bungartz:2013:einfuhrung}. This paper discuses the rendering of objects, which are represented implicitly with mathematical functions, in particular in form of \glspl{sdf} \cite{osher:2006:level}. Lastly, there will be a brief section of how \glspl{sdf} can be manipulated.

% We would like to encourage you to list your keywords within
% the abstract section using the \keywords{...} command.
\keywords{computer graphics, ray tracing, signed distance functions}
\end{abstract}

\glsresetall

\tableofcontents

\section{Introduction}

\section{Literatur}

\subsection{Portals as Occlusion Culling}
potentially visible set
\cite{luebke:1995:portals}
\cite{yang:2014:walkthrough}

\subsection{Portals Algorithms History}


Portal Rendering in SEAMS (\cite{schmalstieg:1999:sewing})

Complex Portal rendering. (I use the same near Buffer technique) \cite{ lowe:2005:technique}


\url{https://th0mas.nl/2013/05/19/rendering-recursive-portals-with-opengl/}
\subsection{Portal Applications}


\subsubsection{VR Interaction 3D Lenses}
\cite{borst:2009:real}
3D lenses which change, the properties. E.g. other rendered objects (x ray lense), inverted colors, fish eye effect.

\subsubsection{Navigation}
Portal Rendering in SEAMS (\cite{schmalstieg:1999:sewing})
\cite{pausch:1995:navigation} navigation with hand held minatures


\subsubsection{Game Mechanik}
Portal, Antichamber, Splitgate Warfare Arena

\subsubsection{VR Movement}

Portal Locomotion in Budget Cuts\url{https://www.youtube.com/watch?v=f786ak3GKQo}

\subsubsection{Fit More Space in Less Space}

\subsubsection{Space Division}
Rooms are split by portals. For each portal only the objects in its room need to be drawn.
\cite{ lowe:2005:technique}

\subsection{Graphics Programming}


\subsubsection{Raytracing}
software,
gpu,
NVIDIA RTX cards, Ray shaders?

\subsubsection{Rasterization}

\subsubsection{GPU Synchronisation}

\subsubsection{Uniforms / Storage Buffer Objects}

\subsubsection{Shaders / (Vulkan-) GLSL / SPIR-V}
branching, fragment discard, early stencil / depth, force early test, Stencil Export

\subsubsection{Occlusion Queries}

\subsubsection{OpenGL}
History

\subsubsection{Vulkan Overview}
input attachments, tile renderers,

(Sub-)Renderpass, Push Constants, Validation Layers, Pipelines, Dynamic state,

Molten VK-Framwork

\subsubsection{Depth / Stencil Test}
Limitations, common uses
Early Z! Fragment discard
Z Fighting,
Depth Formtas
inverse depth buffer with floating point is magic!

combined depth / stencil
\url{https://developer.nvidia.com/content/depth-precision-visualized}


\subsubsection{Instanced Drawing}

\section{Implementation}

\subsection{Software Raytracing}
Even after optimizing, very unlikely to be realtime suitable. 

Include statistics

\subsection{Why Vulkan}
Cross plattform with MoltenVK (apple deprecated openGL)
(Probably) More possibilities due to more explicit compared to OpenGL
More Potential to optimize
Learn effect
Stateless
Validation layers

\subsection{Rendertargets vs Stencil}

\subsection{Generating Camera Matrices}
Explain with images
\subsubsection{Initial Approach}
\subsubsection{Current Approach}

\subsubsection{Further work}
Could be improved. The work to calculate the matrices scales super linear. X^{N}, where X is the portal count and N is the number of iterations. With the use of potentially visible sets, we could dicard combinations which won't be used, reducing the time to calculate the matrices. However, this finding the correct matrix more difficult.


\subsection{Stencil Values}
Problem, value to set stencil must be also be used to compare

use prefixing and compare masks, lessens number of values
start at 1 to avoid ambiguities -> needs more bits, especially problematic for 2 and 4 Portals, which require one more bit than normal.




\subsection{Portal Definition}
Portals are defined as 2 Objects/Endpoints which share a mesh. The both endpoint may have any transformation. From those two endpoints, two teleportation matrices are generated, which when applyied move an object from one end point to the other.

Instead of an mesh representation, other representations, could have been used. E.g. A spehre which is defined by a mathematical equation.

\subsection{Near Buffer}
For only one planar portals a clip plane could work. However we are rendering multiple portals, which are not neccesarily planes.


\subsection{portal z fighting}
when rendering a portal, it is easily z fights with its partner portal, as it would render directly in it. To avoid this we store the winding order as the sign in our rendered depth.  depth is nearly equal and winding order is the same, we have exactly the previously mentionend case and need to discard the fragment. This is easily done by incrasing the comparisions value by some small amout or percentage.

\subsection{Portal Rendering Naive Approach}
One pass per per portal

\subsubsection{Portal drawing orderer Depth first vs breadth first}

breadth first seems better as we only need to clear depth buffer after each layer, because
each draw in a layer will never rendert to the same pixel as another draw in that layer
breadth first allows us to reduce pipelinecount if we use dynamic stencil ref and stencil compare mask

breadth first is easier to stop at a layer

depth first might b

\subsection{Initial Portal Rendering}
In subsequent draws, perform stencil test as well as test against "rendered depth" (similar to clip plane used in Portal 2).
We can force early depth/stencil as we never draw a pixels if it failes the test. We only maybe discard it if the test succeds.

Use pushconstant to decide camera index
clear depth between iterations
use dynamic stencil ref and stencil compare mask, to cut down number of pipelines drastically. Might not be optimal?


\subsubsection{Properties}
fixed number of portals, scales poorly
stencil values known at compile time
Not flexible
Very limited amout of portals an recursions

\subsection{Dynamic Portal Rendering}
Camera matrices are not accessed directly. Instead an an array is accessed which stores the id. Allows to dynamically change camera matrices
Invalid ids means we can skipp that draw -> Set all values in vertex shader to same value to create a degenerate triangle, to improve performance.
Calculate portal Id, dicard if portal id is greater than visible portals
Use portal Id to generate stencil value and the camera id.


\subsubsection{Visible Portal Number algorithm}
in fragmentshader: counter, traverse array, increase for non zeros, ignore at index corresponding to the current portal
write own number or 1 at the ignored location -> Multiple writes of same value should be save, accross "threads", when counting this is skipped so we have no reader at the moment
Needs synchornisation so that value is visible for the next portal!
-> Pseudo occlusion query with immediate result.
-> very conservatic. Portals totaly occluded by later portals or discarded by stencil test still counts as "visible". We can't check for stencil failure, as we can't access the stencil buffer.
-> depth prepass could mitigate this

\subsubsection{Occlusion query alternative}
probably slow
needs to process portals and then again with the result
probably needs some extra throwaway texture

\subsubsection{Properties}
fixed number of visible portals in each portal, which can be dynamic for recursion. E.G inital 4 visible portals, but in each of those only 2 visible portals
Number can be changed at runtime.
Even if visible portal count is too high, we can at least only produce degenerate triangles to save performance.

Requires stencil export (!). Not supported on nvidia cards (source: \url{https://vulkan.gpuinfo.org/listdevices.php?platform=windows&extension=VK_EXT_shader_stencil_export})
No early test for Portal rendering, as we are using stencil export.

\subsubsection{Opportunity: Dynamic Visible Portals count}
If we can detect the initial pass only has few visible portals (e.g. via frustrum culling like techniques), we lower the amount for the initial rendering, and increase it for
the subsequent pass.

Handles case of directly standing in front of a portal!
Have not implemented this

\subsection{Dynamic Portal instance rendering}
We render objects for each rendered portal of previous pass. Only difference is stencil value and camera index.
Camera Index can be calculated with InstanceIndex
Can't change stencil during instance drawing -> Drop Stencil Test. Use extra texture an perform it manually in shader, similar to rendered depth test. 

\subsubsection{Properties}
No early test for Scene.
Recursion and Portal count is only limited by performance, as we can use any number of bits for our test.
Number of draw calls now scales linearly with recursion count (although actuall draw count remains the same).
No Need for Stencil prefixes -> simplifies code, saves bits

Huge Performance Improvement!!! Was 6 times fast in my case. 
(should I measure this and create diagram?)

\subsubsection{Opportunity: Draw Indirect}
Using draw Indirect allows us to only use one draw call per pass, instead of one per object!
Additionally we can manipulate the draw indirect buffer, while calculating the portal id, so that the instance count matches the amount of actually drawn portals.
This way instead of producing degenerate triangles, we don't even draw meshes.
However, could be problematic as we would need to set each value in the shader, having a loop for each scene element, and many writes.

The conditional execution extension could be used (VK\_EXT\_conditional\_rendering) but only to disable drawing entirely, if no portal is rended -> probably not useful.

\subsubsection{Skip instances}
One downside is, that draw instanced draws the object for each portal. However sometimes would could know, that the object is not visible from a specific Portal, e.g. via Potentially visible set tests, or knowing by other means that the portal will never draw that object (e.g. diffrent world layers).
To combat this problem, we could decrease the draw instances count and provide iformation which portals should be skipped.

Instead of using instanceId directly, this skip information will be used to calculate the real id. The rest would behave the same, just using this genreated id instead of instanceId directly.


\subsection{Portal Collision}
On portal collision we apply the same operation on the camera, that would be applied to an object rendered through that portal.
It is implemented by storing a matrix of cumulative portal teleport matrices.
However, it is also possible to apply only a part of it for some interesting effects, but it will result in non seamless translation.

\subsection{Collision Detection (??)}
Raytrace with triangles (could be improved)
KD trees
Surface area heuristic

\subsection{Player Rendering}
When looking through a portal, players might see themselves. Care must be taken when rendering. Standing directly in fron of the portal and touching it slightly could make the player see their only model at their own location. (see valve talk \url{https://www.youtube.com/watch?v=riijspB9DIQ})

\subsection{Watertight Portals}
Objects behind watertight portals, will look as though that portal did not exist. Portals inside on endpoint of an watertight portals, will be visible when looking at the other endpoint, with the teleportation matrix applied. If watertight portals are static, they are essentially useless. The same scene could have been created without them.
However if they change or are created during runtime they can have interesting effects. Such as Scaling objects, or swapping two areas.

\subsection{Face Culling}
Appying face culling only works if no portal changes the scale of an object. Otherwise the triangle order gets changed too and wrong side will be rendered.

\subsection{Level Editor}
Ideally a custom level editor should be built, so the effect of the portals can be seen immediately. However, there was not enough time. After looking for various solution, using the unreal engine seemed best, as the author has some experience with it.

The UE4 Level editor is very convenient and can be customized. The level is exported with a Editor Script. However, some care must be taken as UE4 has a different coordinate system. For Vectors Y and Z must be swapped. For Quaternions, y and z must be swapped and the imaginary part inverted, as we are changing handedness.


\section{Further Work}
\subsection{More than just transforms}
\label{more than transforms}
The current approach uses camera indices, to decide which camera matrix gets applied. However, this approach is not limited to only changing matrices. Inspired by \cite{borst:2009:real} indices could be used to change other paramteres. We could add objects that only render if specific parameters are set, mark pixels for a later post process. Or take as specific branch in any shader.

\subsection{Non-Translating Portals}
The implemented portals have two endpoints. Objects Touch on endpoint, get move to the other. However, it is also possible that the portals don't move the object. In this case there are no two end points. It is just one portal.

Entering the portal would apply one operation (e.g. multiplying by matrix), leaving it applies the inverse operation. Back and Frontface detection needs to be used to decide which operation to apply.


This only works for operations which don't move the possition of an object, unless it is just an rotation and the portal shape looks the same after applying the rotation to it. E.g. a sphere could allow for any rotation. A cube only for rotation in intervals of 90 degrees.
Changing the scale of an objects would also work, but it is important, the the origin remains a the same position. However, this would result in non seamless portal transitions.

Furthermore, more than just transform can be applied. Operations described in the section \ref{more than transforms} would also work.

\subsection{Transparent Objects}
\subsection{Shadows}
Without currect handling, we can see portal locations, as shadows would be cut of.
\subsection{lighting}
Without correct lighting seams would be visible. Proabably very difficult for spot/pointlights. Directional light already works.
\subsection{Collision Detection}
\subsection{Use the Instant Occlusion query in occlusion culling}
we can set conditional rendering paramters during this occlusion culling.
Needs some  recursive portal passes and then one big scene pass, which makes use of conditional rendering, eitehr with the extension or draw indirect and instance counts. Inspired by \cite{yang:2014:walkthrough}














% !TeX spellcheck = en_GB
\section{Level Set Methods and Dynamic Implicit Surfaces}
\cite{osher:2006:level}

In one spatial dimension, a line can be divide in separate pieces using points. For example with the two points  $x = -1$ and $x = +1$ the line is split into the three separate segments.

In one spatial dimension a line can be enclosed by two points to form a sub-domain. For example the two points $x = -1$ and $x = +1$ would split the dimension into three different segments. The segment $(-1, +1)$ represents the inside portion of the domain $\Omega^+$, while the union of other two segments  $(-\infty,-1) \cup (+1,+\infty) $ represents the outside portion $\Omega^-$. The border of the two portions is called interface  $\partial\Omega$. In this example the interface would be represented by the two points $x = -1$ and $x = +1$.

A similar concept can be applied in two dimensions, defining a surface by using lines as interface, as well as in three dimension, defining a volume and using surfaces as interface. Note that the interface is always one less dimension than the sub-domain it encloses.

\subsection{Implicit functions}

While interfaces can be explicitly defined, it is also possible to define them using implicit functions.
A implicit function $\Phi(\vec{x})$ is defined on all points $\vec{x} \in \Re^n$. If $\vec{x}$ lies on the interface the \gls{sdf} returns 0. For points in $\Omega^+$ the result is a negative value, while for $\Omega^-$ it is positive value. Formally:
\begin{itemize}
	\item $\forall \vec{x} \in \partial\Omega,  \Phi(\vec{x}) = 0$ 
	\item $\forall \vec{x} \in \Omega^+,  \Phi(\vec{x}) < 0$
	\item $\forall \vec{x} \in \Omega^-,  \Phi(\vec{x}) > 0$
\end{itemize}

The implicit function version of the previous example would be $\Phi(x) = x^2 - 1$, which gives 0 for the case $x = -1$ and $x = +1$.
In two dimensions the implicit function $\Phi(\vec{x}) = x^2 + y^2 - 1$ defines a circle, while in three dimensions  $\Phi(\vec{x}) = x^2 + y^2 + z^2 - 1$ defines a sphere. $x$, $y$ and $z$ are the respective components of $\vec{x}$.

\subsection{Normal and gradient}

A useful property of an implicit function its gradient. The gradient will always point in the direction where the value of $\Phi$ increases. For a point on the interface this is also the direction of its normal. The gradient $\nabla\phi$ is defined as a vector, the components of which are the resulting value of the partial differential of $\phi$, differentiated in its corresponding dimension. Formally: $\nabla\phi = (\frac{\partial\phi}{\partial x},\frac{\partial\phi}{\partial y},\frac{\partial\phi}{\partial z})$ 

To create the unit normal of a point on the interface we need to normalize the gradient, resulting in the following function: $\vec{N}(\vec{x}) = \frac{\nabla\phi(\vec{x})}{|\nabla\phi(\vec{x})|}$. In addition this does not only work for points on the interface, but for any point, with a few a exceptions where the denominator becomes zero. For $\Phi(x) = x^2 - 1$ and its derivative $\Phi(x)' = 2x$, the normal of $x = 0$ would not be defined.

In cases where the exact derivative cannot be used, it can be approximated. Some examples for approximations are:
\begin{itemize}
	\item forward difference:  $\frac{\partial\phi(\vec{x})}{\partial x} \approx \frac{\phi(x+\Delta x, y, z) - \phi(\vec{x})}{\Delta}$
	\item backward difference: $\frac{\partial\phi(\vec{x})}{\partial x} \approx \frac{\phi(\vec{x}) - \phi(x-\Delta, y, z)}{\Delta}$
	\item central difference: $\frac{\partial\phi(\vec{x})}{\partial x} \approx \frac{\phi(x+\Delta, y, z) - \phi(x-\Delta, y, z)}{2*\Delta}$
\end{itemize}

The versions for the partial derivatives for y, z are analogous to the above examples.


\subsection{Boolean Operations}
\label{section:boolean}
Another useful property of implicit function is that they can easily composed with other implicit functions and Boolean operations. For example the union of the interior regions from two implicit functions $\phi_1(\vec{x})$ and $\phi_2(\vec{x})$ can be described as another implicit function $\phi(\vec{x})$ like this: $\phi(\vec{x}) = min(\phi_1(\vec{x}), \phi_2(\vec{x}))$. Other Boolean operations of the implicit functions' interior regions are:
 
\begin{itemize}
	\item Intersection:  $\phi(\vec{x}) = max(\phi_1(\vec{x}), \phi_2(\vec{x}))$
	\item Counterpart or Negation: $\phi(\vec{x}) = -\phi_1(\vec{x})$ 
	\item Subtracting $\phi_2(\vec{x})$ from $\phi_1(\vec{x})$ :  $\phi(\vec{x}) = max(\phi_1(\vec{x}), -\phi_2(\vec{x}))$
\end{itemize}


\Glspl{sdf} are implicit functions, the absolute value of which is equal to the distance between $\vec{x}$ and it's nearest point on the interface. Formally:  $|\Phi(\vec{x})| = min(|\vec{x} - \vec{y}|) \, \forall \, \vec{y} \in \partial\Omega$. They also posses the property that the length of their gradient is almost always 1, or formally  $|\nabla\phi| = |(\frac{\partial\phi}{\partial x},\frac{\partial\phi}{\partial y},\frac{\partial\phi}{\partial z})| = 1$. However, this is only true for the general case, but not for points which are equidistant to two or more points on the interface.

If the general case is assumed, for a given point $\vec{x}$ the nearest point on the interface $\vec{y}$ can be calculated using the following formula:
$$\vec{y} = \vec{x} - \phi(\vec{x}) * \vec{N}(\vec{x}) $$

This can be simplified as  $\vec{N}(\vec{x}) = \frac{\nabla\phi(\vec{x})}{|\nabla\phi(\vec{x})|}$ and $|\nabla\phi| = 1$, resulting in 
$$\vec{y} = \vec{x} - \phi(\vec{x}) * \nabla\phi(\vec{x}) $$


\section{Ray Marching}
\cite{tuy:1984:direct} \cite{perlin:1989:hypertexture}

Ray marching is an iterative algorithm to find the first intersection, of a ray which is cast into a scene. The algorithm consists of $x_i$ and $\vec{ \Delta x}$, which are the current location and the increment respectively. The initial value for $x_0$ is the rays origin. $\vec{ \Delta x}$ is a fixed vector of a chosen length, which points into rays direction. With each step the algorithm checks whether $x$ is currently inside an object. For instance this could be done with a lookup in a density map or the evaluation of a function. If $x$ is not inside an object increment $x$ by $\vec{ \Delta x}$. Repeat this step until $x$ is inside an object. The intersection lies between the current position  $x_i$ and its previous position  $x_{i-1}$. This process is aborted if a certain amount of iterations is exceeded or $x$ has reached the end of the scene. In this case no intersection was found.

\cite{tuy:1984:direct} \cite{perlin:1989:hypertexture}



\section{Lipschitz continuity, Lipschitz constant and Lipschitz bound}
A function fulfils the Lipschitz continuity if and only if the functions is continuous and the magnitude of its derivative is bounded by a constant value. The smallest possible value for this bound Lipschitz constant for that function. For functions, which contain components with known Lipschitz constants, the Lipschitz constant can be overestimated by the Lipschitz bound. A functions Lipschitz bound is greater or equal than its Lipschitz constant. Lipschitz bound of the sum of two functions is equal or less than the sum of Lipschitz bounds of those functions. 
\cite{hart:1996:sphere} \cite{Heuser:2003}

\section{Sphere tracing}
Sphere tracing can be used to raytrace implicit surfaces, the function of which are Lipschitz continuous. However, the function's derivative does not have to be continuous. It may not even be defined. Sphere tracing finds the first intersection with the implicit surface \cite{hart:1996:sphere}.

Sphere tracing works similar to raymarching. However, instead of a fixed increment it dynamically calculates it using an unbounding sphere. While a bounding sphere would contain an object, an unbounding sphere does not contain any object. The increment is then always set so that its length equals the radius of the unbounding sphere \cite{hart:1996:sphere}.

To calcuate the unbounding sphere the requirements of sphere tracing are used. If the implict surface's function is evaluated a point, then the point is moved by one unit and evaluated again, the two results cannot differ by more than the function's Lipschitz bound. Thus, if we evaluate the implict surface's function a given point and divide the result be the Lipschitz bound, we get the radius of the unbounding sphere. The implicit function's surface function will never evaluate to zero, unless the evaluation point is moved at least by the unbounding sphere's radius. Signed Distance Functions have a Lipschitz Bound of 1. When a Surface is described by Signed Distance Functions, the the unbounding sphere's radius is equal to the function's evaluation result \cite{hart:1996:sphere}.

\section{Signed Distance Function Manipulation}
As described in chapter \ref{section:boolean}, when using \glspl{sdf} a new surface can be created by combining two surfaces using boolean operations. But these are not the only kind of operations that can be used. \textcite{quilez:2008:distfunctions} describes in his article many more operations. This chapter will cover the most significant ones.

\subsection{Rotation, Translation and Scaling}
Rotation can be achieved by rotating the evaluation point by the inverse of the desired rotation. Then then just evaluate the \gls{sdf} as usual. The same process can be used analogously to create translation. Scaling also works the same way. However, after evaluation the resulting value must be multiplied by the scale again to preserve the \gls{sdf}'s properties \cite{quilez:2008:distfunctions}.

\subsection{Repetition}
Repetition is probably one of the most useful operations. A object can be repeated many times without needing to story any extra information. This is done by applying the modulus operator on any of the evaluations points axis, which repeats the object infinitely in that axis \cite{quilez:2008:distfunctions}.

\subsection{Smooth Union, Subtraction and Intersection}
Regular boolean operation do work very well with organic object, as they introduce hard edges. For these objects  \textcite{quilez:2008:distfunctions} recommends using smooth version. They work very similarly to the regular one, but replace $min$ and $max$ functions, by with their smooth versions $smin$ and $smax$. $smin$ behaves (almost) identically to $min$, when the two input values are far apart. However, when the two input parameters are close some kind of smoothing should occur, for example linearly interpolating between the two parameters. There exist multiple possible implementations for $smin$, which have a different speed / quality trade offs. 
\cite{quilez:2008:distfunctions}\cite{quilez:2008:smoothmin}.
\subsection{Rounding}
\subsection{Rounding}















%
% ---- Bibliography ----
%

\printbibliography

\end{document}
