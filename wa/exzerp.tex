% !TeX spellcheck = en_GB
\section{Level Set Methods and Dynamic Implicit Surfaces}
\cite{osher:2006:level}

In one spatial dimension, a line can be divide in separate pieces using points. For example with the two points  $x = -1$ and $x = +1$ the line is split into the three separate segments.

In one spatial dimension a line can be enclosed by two points to form a sub-domain. For example the two points $x = -1$ and $x = +1$ would split the dimension into three different segments. The segment $(-1, +1)$ represents the inside portion of the domain $\Omega^+$, while the union of other two segments  $(-\infty,-1) \cup (+1,+\infty) $ represents the outside portion $\Omega^-$. The border of the two portions is called interface  $\partial\Omega$. In this example the interface would be represented by the two points $x = -1$ and $x = +1$.

A similar concept can be applied in two dimensions, defining a surface by using lines as interface, as well as in three dimension, defining a volume and using surfaces as interface. Note that the interface is always one less dimension than the sub-domain it encloses.

% Implicit functions

While interfaces can be explicitly defined, it is also possible to define them using implicit functions.
A implicit function $\Phi(\vec{x})$ is defined on all points $\vec{x} \in \Re^n$. If $\vec{x}$ lies on the interface the \gls{sdf} returns 0. For points in $\Omega^+$ the result is a negative value, while for $\Omega^-$ it is positive value. Formally:
\begin{itemize}
	\item $\forall \vec{x} \in \partial\Omega,  \Phi(\vec{x}) = 0$ 
	\item $\forall \vec{x} \in \Omega^+,  \Phi(\vec{x}) < 0$
	\item $\forall \vec{x} \in \Omega^-,  \Phi(\vec{x}) > 0$
\end{itemize}

The implicit function version of the previous example would be $\Phi(x) = x^2 - 1$, which gives 0 for the case $x = -1$ and $x = +1$.
In two dimensions the implicit function $\Phi(\vec{x}) = x^2 + y^2 - 1$ defines a circle, while in three dimensions  $\Phi(\vec{x}) = x^2 + y^2 + z^2 - 1$ defines a sphere. $x$, $y$ and $z$ are the respective components of $\vec{x}$.

% normal and gradient
A useful property of an implicit function its gradient. The gradient will always point in the direction where the value of $\Phi$ increases. For a point on the interface this is also the direction of its normal. The gradient $\nabla\phi$ is defined as a vector, the components of which are the resulting value of the partial differential of $\phi$, differentiated in its corresponding dimension. Formally: $\nabla\phi = (\frac{\partial\phi}{\partial x},\frac{\partial\phi}{\partial y},\frac{\partial\phi}{\partial z})$ 

To create the unit normal of a point on the interface we need to normalize the gradient, resulting in the following function: $\vec{N}(\vec{x}) = \frac{\nabla\phi(\vec{x})}{|\nabla\phi(\vec{x})|}$. In addition this does not only work for points on the interface, but for any point, with a few a exceptions where the denominator becomes zero. For $\Phi(x) = x^2 - 1$ and its derivative $\Phi(x)' = 2x$, the normal of $x = 0$ would not be defined.

In cases where the exact derivative cannot be used, it can be approximated. Some examples for approximations are:
\begin{itemize}
	\item forward difference:  $\frac{\partial\phi(\vec{x})}{\partial x} \approx \frac{\phi(x+\Delta x, y, z) - \phi(\vec{x})}{\Delta}$
	\item backward difference: $\frac{\partial\phi(\vec{x})}{\partial x} \approx \frac{\phi(\vec{x}) - \phi(x-\Delta, y, z)}{\Delta}$
	\item central difference: $\frac{\partial\phi(\vec{x})}{\partial x} \approx \frac{\phi(x+\Delta, y, z) - \phi(x-\Delta, y, z)}{2*\Delta}$
\end{itemize}

The versions for the partial derivatives for y, z are analogous to the above examples.


% Boolean operations
Another useful property of implicit function is that they can easily composed with other implicit functions and Boolean operations. For example the union of the interior regions from two implicit functions $\phi_1(\vec{x})$ and $\phi_2(\vec{x})$ can be described as another implicit function $\phi(\vec{x})$ like this: $\phi(\vec{x}) = min(\phi_1(\vec{x}), \phi_2(\vec{x}))$. Other Boolean operations of the implicit functions' interior regions are:
 
\begin{itemize}
	\item Intersection:  $\phi(\vec{x}) = max(\phi_1(\vec{x}), \phi_2(\vec{x}))$
	\item Counterpart or Negation: $\phi(\vec{x}) = -\phi_1(\vec{x})$ 
	\item Subtracting $\phi_2(\vec{x})$ from $\phi_1(\vec{x})$ :  $\phi(\vec{x}) = max(\phi_1(\vec{x}), -\phi_2(\vec{x}))$
\end{itemize}



\Glspl{sdf} are implicit functions, the absolute value of which is equal to the distance between $\vec{x}$ and it's nearest point on the interface. Formally:  $|\Phi(\vec{x})| = min(|\vec{x} - \vec{y}|) \, \forall \, \vec{y} \in \partial\Omega$. They also posses the property that the length of their gradient is almost always 1, or formally  $|\nabla\phi| = |(\frac{\partial\phi}{\partial x},\frac{\partial\phi}{\partial y},\frac{\partial\phi}{\partial z})| = 1$. However, this is only true for the general case, but not for points which are equidistant to two or more points on the interface.

If the general case is assumed, for a given point $\vec{x}$ the nearest point on the interface $\vec{y}$ can be calculated using the following formula:
$$\vec{y} = \vec{x} - \phi(\vec{x}) * \vec{N}(\vec{x}) $$

This can be simplified as  $\vec{N}(\vec{x}) = \frac{\nabla\phi(\vec{x})}{|\nabla\phi(\vec{x})|}$ and $|\nabla\phi| = 1$, resulting in 
$$\vec{y} = \vec{x} - \phi(\vec{x}) * \nabla\phi(\vec{x}) $$


\section{Raytracing}

\section{Ray Marching}
\cite{tuy:1984:direct} \cite{perlin:1989:hypertexture}

Ray marching is an iterative algorithm to find the first intersection, of a ray which is cast into a scene. The algorithm consists of $x_i$ and $\vec{ \Delta x}$, which are to current location and the increment respectively. The initial value $x_0$ is the rays origin. $\vec{ \Delta x}$ is a fixed vector of a chosen length, which points into rays direction. With each step the algorithm checks whether $x$ is currently inside an object. For instance this could be done with a lookup in a density map or the evaluation of a function. If $x$ is not inside an object increment $x$ by $\vec{ \Delta x}$. Repeat this step until $x$ is inside an object. The intersection is lies between the current position  $x_i$ and its previous position  $x_{i-1}$. This process is aborted if a certain amount of iterations is exceeded or $x$ has reached the end of the scene. In this case no intersection was found.

\cite{tuy:1984:direct} \cite{perlin:1989:hypertexture}



\section{Lipschitz continuity, Lipschitz constant and Lipschitz bound}
A function fulfils the Lipschitz continuity if and only if the magnitude of its derivative is bounded by a constant value. The smallest possible value for this bound Lipschitz constant for that function. For functions, which contain components with known Lipschitz constants, the Lipschitz constant can be overestimated by the Lipschitz bound. A functions Lipschitz bound is greater or equal than its Lipschitz constant. Lipschitz bound of the sum of two functions is equal or less than the sum of Lipschitz bounds of those functions. 
\cite{hart:1996:sphere} \cite{Heuser:2003}

\section{Sphere tracing}
\cite{hart:1996:sphere}

Sphere tracing can be used to raytrace surfaces.










