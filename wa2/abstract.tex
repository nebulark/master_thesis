% !TeX spellcheck = en_GB

%The abstract should summarize the contents of the paper
%using at least 70 and at most 150 words. It will be set in 9-point
%font size and be inset 1.0 cm from the right and left margins.
%There will be two blank lines before and after the Abstract. \dots

Ray tracing is a rendering technique, which allows to generate more photorealistic images, when compared with rasterization techniques. Nevertheless, this comes with the cost of a much greater computational expense. In addition, ray tracing allows rendering objects, which are represented by other means than mesh data \cite{bungartz:2013:einfuhrung}. This paper discuses the rendering of objects, which are represented implicitly with mathematical functions, in particular in form of \glspl{sdf} \cite{osher:2006:level}. Lastly, there will be a brief section of how \glspl{sdf} can be manipulated.

% We would like to encourage you to list your keywords within
% the abstract section using the \keywords{...} command.
\keywords{computer graphics, ray tracing, signed distance functions}