% !TeX spellcheck = en_GB

%% constructive solid geometry


\section{Introduction}
\label{section:Introduction}

Ray tracing is an alternative rendering technique to the typically more common Rasterisation variants. Ray tracing enables to generate more photorealistic images, although is comes typically with a much greater computational expense. In addition, due to the nature of its algorithm, ray tracing also allows the rendering objects, which are represented by other means than mesh data \cite{bungartz:2013:einfuhrung}. An implicit surfaces or objects, which are defined with implicit functions, are such an alternative representation. Due to their nature many operation, such as Boolean operations, can be applied. This is either not possible, difficult or computationally expensive with other representations  \cite{osher:2006:level, quilez:2008:distfunctions}. Furthermore, Storing a function instead of mesh data, can save a lot of space. These properties can make them attractive in certain applications. This paper explores the properties of implicit surfaces and \gls{sdf} which are their specialised form. In addition implicit surface rendering and some techniques of manipulating \glspl{sdf} will be discussed.
