% !TeX spellcheck = en_GB

%% constructive solid geometry


\section{Introduction}
\label{section:Introduction}


Ray tracing is a rendering technique, which allows to generate more photorealistic images, when compared with rasterization techniques. Nevertheless, this comes with the cost of a much greater computational expense. However, ray tracing allows the rendering objects, which are represented by other means than mesh data \cite{bungartz:2013:einfuhrung}. An implicitly surface or object, which are defined with implicit functions, is such a alternative representation. Due to their nature many operation, such as boolean operations, can be applied, which are not possible, difficult or computationally expensive with other representations  \cite{osher:2006:level}  \cite{quilez:2008:distfunctions}. Storing a function instead of mesh data, can save a lot of space. These properties can make them attractive in certain applications. This paper explores the properties of implicit surfaces and \gls{sdf} which are their specialised form. In addition implicit surface rendering and some techniques of manipulating \glspl{sdf} will be discussed.
