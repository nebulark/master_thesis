% !TeX spellcheck = en_GB

\section{ideas}
Questions: software rendering techniques, ray tracing, signed distance field

sources: Level Set Methods and Dynamic Implicit Surfaces

Schoolar search terms: 
raytracing


content:
Raytracing using SDF
implict
sdf
raymarching / sphere tracing

raytracing + acceleration (KD-tree)

rasterization

Constructive Solid Geometry

distance fields
adaptively sampled distance fields

\section{Introduction}
\label{section:Introduction}


Ray tracing is a rendering technique, which allows to generate more photorealistic images, when compared with rasterization techniques. Nevertheless, this comes with the cost of a much greater computational expense. However, ray tracing allows the rendering objects, which are represented by other means than explicit vertex data \cite{bungartz:2013:einfuhrung}. This paper discuses the rendering of objects, which are represented implicitly with mathematical functions \cite{osher:2006:level}.  
